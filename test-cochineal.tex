%% -*- coding: utf-8 -*-
\documentclass[twoside,a4page]{article}
\usepackage{polyglossia}
\setdefaultlanguage{polish}
\parindent=9pt
\usepackage{fontspec}
\setmainfont[Ligatures=TeX,Scale=1.20,Numbers=OldStyle]{Cochineal} %% 
\begin{document}
\subsubsection*{\tt 30/36pt}
\fontsize{30pt}{36pt}\selectfont

\begin{tabular}{|c|c|c|c|c|c|c|c|c|c|c|c|c|}\hline
ą & ć & ę & ł & ó & ń & ó & ź & ż & - &  -- &  --- &  ,, \\ \hline
Ą & Ć & Ę & Ł & Ó & Ń & Ó & Ź & Ż & - &  -- &  --- &  '' \\ \hline
\end{tabular}

Małgorzata Skorupa – absolwentka Akademii Muzycznej w~Gdańsku w~klasie
skrzypiec prof.~Henryka Keszkowskiego. Doskonaliła swoje umiejętności
na kursach w~Polsce, Austrii, Niemczech i~Węgrzech. Małgorzata Skorupa
koncertuje jako solistka i~kameralistka. Dokonała wielu prawykonań
dzieł współczesnych kompozytorów. Posiada bardzo wszechstronny
repertuar (od baroku do współczesności). Współpracuje z~orkiestrami
symfonicznymi i~kameralnymi. Jest profesorem Akademii Muzycznej
w~Gdańsku oraz Ogólnokształcącej Szkoły Muzycznej I~i II~stopnia
w~Gdańsku. Jej uczniowie i~studenci byli wielokrotnie laureatami
konkursów skrzypcowych w~Polsce i~za granicą. Pełni funkcję prorektora
do spraw studenckich.

\subsubsection*{\tt 18/24pt}
\fontsize{24pt}{30pt}\selectfont

Anna Sawicka – absolwentka Akademii Muzycznej Gdańsku w~klasie
wiolonczeli Romana Sucheckiego oraz Szkoły Muzyki Dawnej w~Genewie
w~klasie wiolonczeli barokowej Phillippe Mermoude'a i~violi da gamba
Roberto Gini. Swoje umiejętności z~zakresu wykonawstwa muzyki dawnej
poszerzała na wielu kursach w~Polsce, we Włoszech i~Szwajcarii
u~takich mistrzów jak Wieland Kuijken, Andrew Manze, Rinaldo
Alessandrini, Enrico Gatti, Chiara Banchini, Richte van der Meer.
Ma w~swoim repertuarze najdawniejsze utwory (XVII i~XVIII–wieczne) na
wiolonczelę solo. Jest członkiem założycielem Polskiego Towarzystwa
Przyjaciół Muzyki Dawnej. Prowadzi ożywioną działalność jako solistka
i~kameralistka – koncerty w~Polsce, Francji, Szwajcarii, Niemczech, we
Włoszech i~USA, a~także zajmuje się pedagogiką i~pracą naukową
dotyczącą historii wiolonczeli. W~latach 1992–93 była
koncertmistrzem Państwowej Orkiestry Kameralnej w~Toruniu, w~1996
objęła stanowisko koncertmistrza orkiestry Państwowej Opery Bałtyckiej
w~Gdańsku. Współpracuje z~wydawnictwem Eufonium, przygotowując do
druku zapomniane utwory wiolonczelowe. Jest animatorką życia
muzycznego w~Sopocie – cykle Świętojerskie Spotkania Muzyczne
oraz Muzyczne Wieczory u~św.~Andrzeja Boboli.

\subsubsection*{\tt 12/18pt}
\fontsize{12pt}{18pt}\selectfont

Elżbieta Rosińska – absolwentka Akademii Muzycznej im.~Stanisława
Moniuszki w~Gdańsku w~klasie akordeonu Józefa
Madanowskiego. Uczestniczyła w~wielu kursach mistrzowskich w~kraju
i~zagranicą, doskonaląc swoje umiejętności pod kierunkiem tak
znakomitych artystów jak Mogens Ellegaard, Stefan Hussong, Mie~Miki,
Friedrich Lips, Matti Rantanen. Jako solistka i~kameralistka
koncertowała w~Polsce, Niemczech, Finlandii, Słowacji, Austrii,
Iranie oraz Szwajcarii. Dokonała kilkunastu prawykonań utworów, m.in.
K.~Olczaka, M.~Gordiejuka, K.~Naklickiego, P.~Słopeckiego. Prowadzi kursy
i~seminaria dla nauczycieli gry na akordeonie. Jest autorką artykułów
poświęconych muzyce akordeonowej i~jej wirtuozom, zamieszczonych na
łamach ,,Ruchu Muzycznego'', ,,Przeglądu Muzycznego'', ,,Twojej Muzy''
oraz czasopism zagranicznych. Opublikowała książkę Polska literatura
akordeonowa 1955–1996. Prowadzi internetowy Katalog Polskiej Muzyki
Akordeonowej. Od 20~lat organizuje letnie kursy dla młodych
akordeonistów w~Wejherowie – Kaszubskie Warsztaty Akordeonowe. Pracuje
w~Akademii Muzycznej w~Gdańsku na stanowisku profesora, pełni funkcję
dziekana Wydziału Instrumentalnego.


Ewa Nowacka – ukończyła studia instrumentalne w~Gdańsku oraz
w~Hamburgu. Ważnym elementem na jej drodze rozwoju był udział w~wielu
kursach mistrzowskich. Podczas studiów w~Hamburgu odkryła możliwości
klarnetu basowego.  Umiejętności w~grze na tym instrumencie
doskonaliła podczas kursów z~Jean-Marc Fessardem oraz Henry
Bokiem. Ewa Nowacka regularnie występuje w~projektach muzycznych
w~Hamburgu i~innych miastach, odbyła również dwukrotnie tournee
orkiestrowe w~Chinach. Od 2011~roku uczy klarnetu w~\emph{Staatliche
Jugendmusikschule Hamburg}, jest jurorem w~konkursie \emph{Jugend musiziert\/}
oraz dyrektorem artystycznym konkursu \emph{Jugend musiziert Hamburg
  Nord}. Obecnie jest uczestniczką studiów doktoranckich na AM
w~Gdańsku. Tematem dysertacji jest literatura na klarnet basowy.

\subsubsection*{\tt 10/14pt}

\fontsize{10pt}{14pt}\selectfont
Robert Horna – ceniony i~uznany solista i~kameralista, zaliczany do
czołówki polskich gitarzystów klasycznych. Jest absolwentem Akademii
Muzycznej we Wrocławiu w klasie gitary prof.~Piotra Zaleskiego oraz
Konserwatorium Muzycznego w~Enschede w~Holandii. Studiował również
jazz u~znakomitego polskiego gitarzysty Artura Lesickiego. Podróże
koncertowe wiodły go przez niemal całą Europę oraz Amerykę
Południową. Regularnie prowadzi klasy mistrzowskie i~zasiada w~jury
festiwali i~konkursów gitarowych o~międzynarodowym zasięgu. Robert
Horna koncertuje jako solista, kameralista oraz wykonuje koncerty
z~towarzyszeniem orkiestr symfonicznych. Jego bogaty repertuar obejmuje
utwory muzyki dawnej, dzieła klasyków, kompozycje współczesne
a~muzyczne zainteresowania sięgają rożnych stylów. Na stałe współpracuje
z~czołowymi artystami muzyki klasycznej i~jazzowej, między innymi:
Krzysztof Pełech – gitara, Izabella Kopeć – mezzosopran, Klaudiusz
Baran i~Wiesław Prządka – bandoneon, Janusz Strobel – jazzowa gitara
klasyczna. Ma na swoim koncie trzy solowe płyty: „Milonga del Angel”
(CD Accord 2000), „Las Cuatro Estaciones Porteñas” (CD Accord 2005)
oraz „Suity barokowe” (2009). Jest autorem aranżacji i~transkrypcji na
gitarę, docenionych zarówno przez środowisko gitarowe jak i~przez
krytyków muzycznych. Od 2013 roku prowadzi klasę gitary w~Akademii
Muzycznej im.~Stanisława Moniuszki w~Gdańsku.


\end{document}
